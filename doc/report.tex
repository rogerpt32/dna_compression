\documentclass[conference,a4paper]{IEEEtran}
\usepackage[utf8]{inputenc}
\usepackage{booktabs}
\usepackage{enumitem} %resume enumerate
\usepackage{graphicx}
\usepackage{caption}
\usepackage{subcaption}
\usepackage{hyperref}
\usepackage{url}
\usepackage[x11names, svgnames, rgb]{xcolor}
\usepackage{tikz}
\usetikzlibrary{snakes,arrows,shapes}
\usepackage{amsmath}

\def\BibTeX{{\rm B\kern-.05em{\sc i\kern-.025em b}\kern-.08em
    T\kern-.1667em\lower.7ex\hbox{E}\kern-.125emX}}

\begin{document}

\title{Human Activity Recognition Using Smartphones}

\author{\IEEEauthorblockN{Roger Pujol}
\IEEEauthorblockA{\textit{Universitat Politècnica de Catalunya (UPC)}\\
Barcelona, Spain \\
roger.pujol.torramorell@est.fib.upc.edu}}

\date{\today}

\maketitle

\begin{abstract}
Abstract goes here.

The code of this project is Open Source and can be found in: \url{https://github.com/rogerpt32/dna_compression} \cite{code}.
\end{abstract}

\section{Introduction}
The smartphones have become probably the most used device for almost everybody. Furthermore everybody carries their smartphone everywhere they go and these devices are usually full of sensors. This makes the smartphone a perfect device to get data from people. In this paper we will use the data from the accelerometers and gyroscopes of a smartphone, in order to recognize what is doing the person carrying the device. Since this is a project for a Data Mining course, we will focus in testing several classification methods and comparing them for this particular data.

\section{Summary}
The summary of the paper(s), data structure(s) (10\%),

\section{Experiments}

\end{document}